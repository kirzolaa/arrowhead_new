\documentclass{article}
\usepackage{amsmath}
\usepackage{amssymb}
\usepackage{graphicx}

\title{Notes on Berry Phase Calculation and Analysis}
\date{\today}

\begin{document}

\maketitle

\section{Berry Phase Calculation}

The Berry phase is a geometric phase acquired by a quantum system evolving adiabatically along a closed path in its parameter space. It is calculated by integrating the Berry connection over the closed path.

\subsection{Berry Connection}

The Berry connection, denoted as $A_n(\mathbf{R})$, is defined as:

$$
A_n(\mathbf{R}) = \langle n(\mathbf{R}) | i \nabla_\mathbf{R} | n(\mathbf{R}) \rangle
$$

where $|n(\mathbf{R})\rangle$ is the $n$-th eigenstate of the system's Hamiltonian, and $\nabla_\mathbf{R}$ is the gradient operator with respect to the parameter vector $\mathbf{R}$.

\subsection{Berry Phase}

The Berry phase, denoted as $\gamma_n$, for a closed path $C$ in the parameter space is given by:

$$
\gamma_n = \oint_C A_n(\mathbf{R}) \cdot d\mathbf{R}
$$

\section{Numerical Calculation of Berry Phase}

In numerical calculations, the Berry connection and Berry phase are approximated using discrete methods.

\subsection{Discrete Berry Connection}

The Berry connection can be approximated using a finite difference scheme. For example, using a forward difference:

$$
A_n(\mathbf{R}_i) \approx \left\langle n(\mathbf{R}_i) \middle| i \frac{|n(\mathbf{R}_{i+1})\rangle - |n(\mathbf{R}_i)\rangle}{\mathbf{R}_{i+1} - \mathbf{R}_i} \right\rangle
$$

where $\mathbf{R}_i$ are discrete points along the path in parameter space.

\subsection{Discrete Berry Phase}

The Berry phase is then approximated by summing the Berry connection along the discrete path:

$$
\gamma_n \approx \sum_i A_n(\mathbf{R}_i) \cdot (\mathbf{R}_{i+1} - \mathbf{R}_i)
$$

\section{Berry Curvature}

The Berry curvature, denoted as $\mathbf{\Omega}$, is a fundamental quantity associated with the Berry connection. It measures the "twisting" or non-trivial geometric structure of the eigenstates in the parameter space.

\subsection{Definition}

In a general parameter space with multiple parameters, the Berry curvature is defined as the curl of the Berry connection:

$$
\mathbf{\Omega} = \nabla \times \mathbf{A}
$$

where $\mathbf{A}$ is the Berry connection.

\subsection{Calculation in 1D Parameter Space}

In the specific case where the parameter space is one-dimensional (e.g., parameterized by a single parameter $\theta$), the curl simplifies, and the Berry curvature is related to the derivative of the Berry connection with respect to the parameter.

If $A_\theta$ is the Berry connection with respect to the parameter $\theta$, and $R$ represents the parameter vector, then the Berry curvature can be approximated as:

$$
\Omega \approx \frac{dA}{dR} \approx \frac{dA_\theta}{d\theta} \frac{d\theta}{dR_\theta}
$$

\subsection{Numerical Approximation}

In numerical calculations, the derivative is approximated using finite difference methods. For example, using a numerical derivative of the Berry connection with respect to $\theta$ and relating it to the derivative with respect to $R$:

$$
\Omega_i \approx \frac{A_{\theta, i+1} - A_{\theta, i}}{\theta_{i+1} - \theta_i} \left( \frac{dR_{\theta, i}}{d\theta} \right)^{-1}
$$

Where:
\begin{itemize}
\item $\Omega_i$ is the Berry curvature at the $i$-th point.
\item $A_{\theta, i}$ is the Berry connection with respect to $\theta$ at the $i$-th point.
\item $\theta_i$ is the value of the parameter $\theta$ at the $i$-th point.
\item $\frac{dR_{\theta, i}}{d\theta}$ is the derivative of the $i$-th parameter vector $R_{\theta, i}$ with respect to $\theta$.
\end{itemize}

\subsection{Shape of the Berry Curvature Output}

The Berry curvature, when calculated numerically as described above, is typically represented as a 2D array with a shape of (number of points, number of eigenstates).

\begin{itemize}
\item $\textbf{Shape:}$ (number of points, number of eigenstates)
\item For example, if you have calculated the Berry curvature at 5000 different points in your parameter space for a system with 4 eigenstates, the Berry curvature array will have a shape of (5000, 4).
\end{itemize}

This shape is important because:

\begin{itemize}
\item The Berry curvature is not a single value for each eigenstate. It's a $\textbf{local}$ property that can vary as you move through the parameter space.
\item $\text{berry\_curvature}[i, n]$ represents the Berry curvature for the $\textbf{n}$-th eigenstate at the $\textbf{i}$-th point along the path in your parameter space.
\end{itemize}

\subsection{Interpretation}

The Berry curvature is a measure of the non-Abelian phase acquired by a quantum system. A non-zero Berry curvature indicates that the eigenstates of the system change in a complex and non-trivial manner as the parameters of the system are varied. It plays a crucial role in various physical phenomena, including topological effects in condensed matter physics.

\section{Code Implementation}

The provided Python code implements the calculation of the Berry phase for a given Hamiltonian, parameter space, and eigenstates. It includes two methods for calculating the Berry phase:

\begin{enumerate}
    \item  \textbf{Original Method:} This method calculates the Berry connection using a finite difference approximation with respect to the parameter vector $\mathbf{R}$ and then integrates it to obtain the Berry phase.
    \item  \textbf{Theta Derivative Method:} This method calculates the Berry connection with respect to a parameter $\theta$ that parameterizes the path in $\mathbf{R}$ space. It then attempts to relate this to the Berry connection in the $\mathbf{R}$ space.
\end{enumerate}

\section{Analysis of Results}

It's important to clarify that "trivial" and "non-trivial" refer to the nature of the Berry phase itself, not to separate types of Berry phases that exist independently. A Berry phase is a geometric phase acquired by a quantum system evolving adiabatically along a closed path in its parameter space. \cite{berry1984}

If the Berry phase calculation consistently yields a trivial result (i.e., $\gamma_n = 0$ or a multiple of $2\pi$), it suggests that the system, for the chosen parameters and the path in parameter space, does not exhibit non-trivial topological features.

\subsection{Possible Reasons for Trivial Berry Phase}

\begin{itemize}
    \item   \textbf{System Properties:} The system itself might have a trivial Berry phase due to its inherent symmetries or lack of topological structure.
    \item   \textbf{Parameter Choice:} The chosen parameters for the Hamiltonian might not be in a regime where non-trivial topological phenomena occur.
    \item   \textbf{Path in Parameter Space:} The chosen path might not enclose any regions in the parameter space where non-trivial Berry curvature exists.
    \item   \textbf{Numerical Precision:} In some cases, numerical limitations or approximations might lead to a trivial result, especially for very small paths or rapidly oscillating Berry curvature.
\end{itemize}

\subsection{Strategies to Explore Non-Trivial Berry Phase}

There isn't a separate calculation to perform to find the "trivial part" because the result you're getting is the Berry phase for the system under those conditions.

However, if you want to explore the possibility of a non-trivial Berry phase in your system, you need to modify the conditions and recalculate. Here's what you can do:

\begin{itemize}
    \item   \textbf{Vary System Parameters:} Vary the parameters of your Hamiltonian (e.g., $aVx$, $aVa$, $\omega$, $c_{const}$, $x_{shift}$). Some parameter combinations might lead to topological features (like avoided crossings or degeneracies) that give rise to a non-trivial Berry phase.
    \item   \textbf{Modify the Path:} Instead of a simple circle, try different closed paths in your parameter space. For example:
        \begin{itemize}
            \item   Elliptical paths
            \item   Square paths
            \item   Paths that enclose different regions in the parameter space. The Berry phase is path-dependent, so changing the path can reveal non-trivial topology that a circular path might miss. \cite{nakahara2003}
        \end{itemize}

    \item   \textbf{Analyze Hamiltonian and Eigenvalues:} Analyze your Hamiltonian matrices and the resulting eigenvalues. Look for:
        \begin{itemize}
            \item   \textbf{Degeneracies:} Points in the parameter space where two or more eigenvalues are equal.
            \item   \textbf{Avoided Crossings:} Regions where eigenvalues come close together but don't actually cross. These features are often associated with non-trivial Berry phases. \cite{hatsugai2005}
        \end{itemize}
        If your system does not exhibit these features for the chosen parameters and path, it's less likely to have a non-trivial Berry phase.
    \item   \textbf{Consider a Simplified Model:} If your current model is complex, try to simplify it to a system where you know that a non-trivial Berry phase exists. For example, consider a 2x2 system with a known topological structure, like a system describing a Dirac point. If you can calculate a non-trivial Berry phase for the simplified model, it will help you validate your code and approach.
\end{itemize}

In summary: You don't calculate the "trivial part." You modify the system or the way you probe it (parameters, path) to see if you can reveal non-trivial behavior. If, after extensive exploration, you still only find zero Berry phase, then that is a property of the system under those conditions.

\section{Eigenvector Visualization}

To better understand the behavior of the eigenstates, it is helpful to visualize their components as a function of the parameter (e.g., $\theta$). Plotting the real and imaginary parts of the eigenvector components can reveal important information about their evolution along the path in parameter space.

\section{References}

\begin{thebibliography}{9}
    \bibitem{berry1984}
      M.~V.~Berry,
      \emph{Quantal phase factors accompanying adiabatic changes},
      Proc. R. Soc. Lond. A \textbf{392}, 45-57 (1984).

    \bibitem{nakahara2003}
      M.~Nakahara,
      \emph{Geometry, Topology and Physics},
      Graduate Student Series in Physics,
      2nd ed.,
      IOP Publishing, Bristol (2003).

    \bibitem{hatsugai2005}
      Y.~Hatsugai,
      \emph{Topological aspects of quantum transport phenomena},
      Lecture Notes in Physics \textbf{669}, 37-100 (2005).
\end{thebibliography}

\end{document}