\documentclass{article}
\usepackage{amsmath}
\usepackage{amssymb}
\usepackage{bm}

\begin{document}

\section{Hamiltonian Formulation}
The Hamiltonian $H(\bm{R}_{\theta})$ is a 4x4 arrowhead matrix defined as:
\begin{equation}
H(\bm{R}_{\theta}) = \begin{pmatrix}
E_0(R_{\theta,i}) & c_{10} & c_{20} & c_{30} \\
c_{10} & E_1(R_{\theta,i}) & 0 & 0 \\
c_{20} & 0 & E_2(R_{\theta,i}) & 0 \\
c_{30} & 0 & 0 & E_3(R_{\theta,i})
\end{pmatrix}
\end{equation}
where $E_0(R_{\theta,i}) = \hbar\omega + \sum_{i=0}^2 V_x^{(i)}(R_{\theta,i})$ and $E_{i+1}(R_{\theta,i}) = E_0(R_{\theta,i}) + V_a^{(i)}(R_{\theta,i}) - V_x^{(i)}(R_{\theta,i})$ for $i=0,1,2$.

Here $\omega$ is a frequency parameter, and $V_x$ and $V_a$ are potential terms that depend on the parameter vector $R(\theta)$. These potential terms are defined as follows:
\begin{align}
V_x^{(i)}(R_{\theta,i}) &= a \cdot (R_{\theta,i})^2 + c \\
V_a^{(i)}(R_{\theta,i}) &= a \cdot (R_{\theta,i} - x_{\text{shift}})^2 + c
\end{align}

The coupling constants $c_{10}$, $c_{20}$, and $c_{30}$ are calculated using the transitional dipole moment between the ground state and excited states. The transitional dipole moment is computed as:
\begin{equation}
c_{i0} = c_{0i} = \langle \psi_i | \hat{\bm{r}} | \psi_0 \rangle
\end{equation}
where $\psi_0$ is the ground state eigenvector, $\psi_i$ is the $i^{th}$ excited state eigenvector, and $\hat{\bm{r}}$ is the position operator. These couplings are then used in the Hamiltonian matrix:
\begin{equation}
H(\bm{R}_{\theta}) = \begin{pmatrix}
E_0(R_{\theta,i}) & c_{10} & c_{20} & c_{30} \\
c_{10} & E_1(R_{\theta,i}) & 0 & 0 \\
c_{20} & 0 & E_2(R_{\theta,i}) & 0 \\
c_{30} & 0 & 0 & E_3(R_{\theta,i})
\end{pmatrix}
\end{equation}

\section{$\bm{R}_{\theta}$ generation}
The $\bm{R}_{\theta}$ vector traces a perfect circle orthogonal to the $x=y=z$ line using the \texttt{create\_perfect\_orthogonal\_vectors} function from the Arrowhead/generalized package.


\section{Berry Connection}
The Berry connection $A(\bm{R}_{\theta})$ is calculated using:
\begin{equation}
A_{n}(\bm{R}_{\theta}) = \langle n(\bm{R}_{\theta}) | i \partial_{\bm{R}_{\theta}} | n(\bm{R}_{\theta}) \rangle
\end{equation}
where $|n(\bm{R}_{\theta})\rangle$ are the eigenstates of $H(\bm{R}_{\theta})$.

\section{Berry Phase}
The Berry phase $\gamma_n$ for state $n$ is obtained by integrating the Berry connection:
\begin{equation}
\gamma_n = \int_0^{2\pi} A_n(\bm{R}_{\theta}) d\bm{R}_{\theta}
\end{equation}

\section{Verification}
We verify the eigenvalue equation $H(\bm{R}_{\theta})|n(\bm{R}_{\theta})\rangle = E_n(\bm{R}_{\theta})|n(\bm{R}_{\theta})\rangle$ by comparing:
\begin{equation}
H(\bm{R}_{\theta})|n(\bm{R}_{\theta})\rangle \quad \text{vs} \quad E_n(\bm{R}_{\theta})|n(\bm{R}_{\theta})\rangle
\end{equation}

\section{Visualization}
For each state $n$, we plot:
\begin{itemize}
\item Magnitude of $H(\bm{R}_{\theta})|n(\bm{R}_{\theta})\rangle$ and $E_n(\bm{R}_{\theta})|n(\bm{R}_{\theta})\rangle$
\item Real and imaginary components separately
\item All four vector components
\end{itemize}

\section{References}
\begin{itemize}
\item M. V. Berry, \textit{Quantal phase factors accompanying adiabatic changes}, Proc. R. Soc. Lond. A 392, 45-57 (1984)
\item D. J. Thouless, \textit{Topological Quantum Numbers in Nonrelativistic Physics}, World Scientific (1998)
\item B. Simon, \textit{Holonomy, the Quantum Adiabatic Theorem, and Berry's Phase}, Phys. Rev. Lett. 51, 2167 (1983)
\item J. E. Avron, R. Seiler, L. G. Yaffe, \textit{Adiabatic Theorems and Applications to the Quantum Hall Effect}, Commun. Math. Phys. 110, 33-49 (1987)
\end{itemize}

\end{document}
