\newpage
\section{Usage Examples}

This section provides examples of how to use the Generalized Arrowhead Framework. It includes examples of using the package as a Python module and as a command-line tool, with a focus on both vector generation and arrowhead matrix analysis.

\subsection{Basic Usage as a Python Module}

The following example shows how to use the package as a Python module to generate and visualize orthogonal vectors with default parameters:

\begin{lstlisting}[language=Python]
import numpy as np
from generalized import create_orthogonal_vectors, plot_vectors

# Generate orthogonal vectors with default parameters
# (origin at [0, 0, 0], d=1.0, theta=pi/4)
vectors = create_orthogonal_vectors(origin=[0, 0, 0])

# Plot the vectors in 3D with enhanced visualization
plot_vectors(vectors, origin=[0, 0, 0], enhanced_visualization=True, 
           axis_colors=['r', 'g', 'b'], show_coordinate_labels=True, 
           equal_aspect_ratio=True)
\end{lstlisting}

\subsection{Customizing Vector Generation}

The following example shows how to customize the vector generation by specifying the origin, distance parameter, and angle parameter:

\begin{lstlisting}[language=Python]
import numpy as np
import math
from generalized import create_orthogonal_vectors, plot_vectors

# Generate orthogonal vectors with custom parameters
origin = [1, 1, 1]
d = 2.0
theta = math.pi / 3

vectors = create_orthogonal_vectors(origin=origin, d=d, theta=theta)

# Plot the vectors in 3D
plot_vectors(vectors, origin=origin, title=f"Orthogonal Vectors (Origin={origin}, d={d}, theta={theta})")
\end{lstlisting}

\subsection{Using the VectorConfig Class}

The following example shows how to use the \texttt{VectorConfig} class to configure vector generation and visualization:

\begin{lstlisting}[language=Python]
import numpy as np
import math
from generalized import create_orthogonal_vectors, plot_vectors, VectorConfig

# Create a configuration
config = VectorConfig(
    origin=[0, 0, 2],
    d=1.5,
    theta=math.pi / 6,
    plot_type="2d",
    title="Custom Configuration",
    save_path="custom_config.png"
)

# Generate orthogonal vectors using the configuration
vectors = create_orthogonal_vectors(
    origin=config.origin,
    d=config.d,
    theta=config.theta
)

# Plot the vectors using the configuration
plot_vectors(
    vectors,
    origin=config.origin,
    plot_type=config.plot_type,
    title=config.title,
    show_plot=config.show_plot,
    save_path=config.save_path
)
\end{lstlisting}

\subsection{Saving and Loading Configurations}

The following example shows how to save a configuration to a file and load it later:

\begin{lstlisting}[language=Python]
import numpy as np
import math
from generalized import VectorConfig, create_orthogonal_vectors, plot_vectors

# Create a configuration
config = VectorConfig(
    origin=[0, 0, 2],
    d=1.5,
    theta=math.pi / 6,
    plot_type="2d",
    title="Custom Configuration"
)

# Save the configuration to a file
config.save_to_file("config.json")

# Later, load the configuration from the file
loaded_config = VectorConfig.load_from_file("config.json")

# Generate orthogonal vectors using the loaded configuration
vectors = create_orthogonal_vectors(
    origin=loaded_config.origin,
    d=loaded_config.d,
    theta=loaded_config.theta
)

# Plot the vectors using the loaded configuration
plot_vectors(
    vectors,
    origin=loaded_config.origin,
    plot_type=loaded_config.plot_type,
    title=loaded_config.title,
    show_plot=loaded_config.show_plot,
    save_path=loaded_config.save_path
)
\end{lstlisting}

\subsection{Checking Orthogonality}

The following example shows how to check if a set of vectors is orthogonal:

\begin{lstlisting}[language=Python]
import numpy as np
from generalized import create_orthogonal_vectors, check_orthogonality

# Generate orthogonal vectors
vectors = create_orthogonal_vectors(origin=[0, 0, 0])

# Check if the vectors are orthogonal
is_orthogonal = check_orthogonality(vectors, origin=[0, 0, 0])

print(f"Vectors are orthogonal: {is_orthogonal}")
\end{lstlisting}

\subsection{Perfect Orthogonal Circle Generation}

The following example shows how to generate a perfect circle in the plane orthogonal to the x=y=z line with enhanced visualization features:

\begin{lstlisting}[language=Python]
import numpy as np
import matplotlib.pyplot as plt
from mpl_toolkits.mplot3d import Axes3D
from generalized import create_orthogonal_vectors
from generalized.visualization import setup_enhanced_3d_axes

# Set parameters
R_0 = np.array([1, 2, 3])  # Origin
d = 2.0                    # Distance parameter
num_points = 36            # Number of points

# Generate a perfect circle
vectors = create_orthogonal_vectors(R_0, d, num_points, perfect=True)

# Verify properties
distances = np.array([np.linalg.norm(v - R_0) for v in vectors])
unit_111 = np.array([1, 1, 1]) / np.sqrt(3)
dot_products = np.array([np.abs(np.dot(v - R_0, unit_111)) for v in vectors])

print(f"Mean distance from origin: {np.mean(distances)}")
print(f"Standard deviation of distances: {np.std(distances)}")
print(f"Maximum dot product with (1,1,1): {np.max(dot_products)}")

# Create 3D visualization with enhanced features
fig = plt.figure(figsize=(10, 8))
ax = fig.add_subplot(111, projection='3d')

# Apply enhanced axis styling with custom colors and labels
setup_enhanced_3d_axes(ax, vectors, axis_colors=['r', 'g', 'b'], 
                     show_coordinate_labels=True, equal_aspect_ratio=True, 
                     buffer_factor=0.1)

# Plot the circle
ax.scatter(vectors[:, 0], vectors[:, 1], vectors[:, 2], label='Perfect Circle')

# Plot the origin
ax.scatter(R_0[0], R_0[1], R_0[2], color='red', s=100, marker='o', label='Origin R_0')

# Plot the x=y=z line
line = np.array([[-1, -1, -1], [7, 7, 7]])
ax.plot(line[:, 0], line[:, 1], line[:, 2], 'r--', label='x=y=z line')

ax.set_title('Perfect Circle Orthogonal to x=y=z Line')
ax.legend()

plt.show()
\end{lstlisting}

\subsection{Generating Circle Segments}

The following example shows how to generate circle segments by specifying start and end angles:

\begin{lstlisting}[language=Python]
import numpy as np
import matplotlib.pyplot as plt
from mpl_toolkits.mplot3d import Axes3D
from generalized import create_orthogonal_vectors

# Set parameters
R_0 = np.array([1, 2, 3])  # Origin
d = 2.0                    # Distance parameter
num_points = 18            # Number of points

# Define theta ranges
theta_ranges = [
    (0, np.pi/2),          # Quarter circle
    (0, np.pi),            # Half circle
    (np.pi/4, 3*np.pi/4),  # Middle segment
    (0, 2*np.pi)           # Full circle
]

# Create 3D visualization with enhanced features
fig = plt.figure(figsize=(10, 8))
ax = fig.add_subplot(111, projection='3d')

# Collect all vectors for proper axis scaling
all_vectors = []
for start_theta, end_theta in theta_ranges:
    # Generate vectors for this segment
    vectors = create_orthogonal_vectors(R_0, d, num_points, perfect=True, 
                                      start_theta=start_theta, end_theta=end_theta)
    all_vectors.append(vectors)

# Apply enhanced axis styling with all points for proper scaling
all_points = np.vstack(all_vectors)
from generalized.visualization import setup_enhanced_3d_axes
setup_enhanced_3d_axes(ax, all_points)

# Plot each circle segment
for i, ((start_theta, end_theta), vectors) in enumerate(zip(theta_ranges, all_vectors)):
    # Plot the circle segment
    range_desc = f"({start_theta:.2f}, {end_theta:.2f})"
    ax.scatter(vectors[:, 0], vectors[:, 1], vectors[:, 2], label=f'$\\theta \\in$ {range_desc}')

# Plot the origin
ax.scatter(R_0[0], R_0[1], R_0[2], color='red', s=100, marker='o', label='Origin R_0')

# Plot the x=y=z line
line = np.array([[-1, -1, -1], [7, 7, 7]])
ax.plot(line[:, 0], line[:, 1], line[:, 2], 'r--', label='x=y=z line')

ax.set_title('Perfect Circle Segments with Different Theta Ranges')
ax.legend()

plt.show()
\end{lstlisting}

\subsection{Using the Command-line Interface}

The command-line interface provides access to both vector generation and arrowhead matrix analysis functionality through the \texttt{main.py} script.

\subsubsection{Vector Generation Examples}

\paragraph{Basic Vector Usage}

\begin{lstlisting}[language=bash]
python -m generalized.main vector
\end{lstlisting}

This command generates and visualizes orthogonal vectors with default parameters (origin at [0, 0, 0], d=1.0, theta=pi/4).

\paragraph{Customizing Vector Generation}

\begin{lstlisting}[language=bash]
python -m generalized.main vector --origin 1 1 1 --d 2.0 --theta 1.047
\end{lstlisting}

This command generates and visualizes orthogonal vectors with custom parameters (origin at [1, 1, 1], d=2.0, theta=pi/3).

\paragraph{Customizing Visualization}

\begin{lstlisting}[language=bash]
python -m generalized.main vector --plot-type 2d --title "Custom Visualization" --save-path custom.png
\end{lstlisting}

This command generates orthogonal vectors with default parameters and visualizes them with custom visualization options (2D plot, custom title, save to file).

\paragraph{Generating Perfect Orthogonal Circles}

\begin{lstlisting}[language=bash]
python -m generalized.main vector --origin 1 2 3 --d 2.0 --theta-range 0 36 6.28 --perfect
\end{lstlisting}

This command generates a perfect circle in the plane orthogonal to the x=y=z line with 36 points, centered at [1, 2, 3] with a distance of 2.0.

\paragraph{Generating Circle Segments}

\begin{lstlisting}[language=bash]
python -m generalized.main vector --origin 1 2 3 --d 2.0 --theta-range 0 18 3.14159 --perfect
\end{lstlisting}

This command generates a half-circle segment (from 0 to $\pi$) in the plane orthogonal to the x=y=z line.

\paragraph{Using a Configuration File}

\begin{lstlisting}[language=bash]
python -m generalized.main vector --config config.json
\end{lstlisting}

This command loads a configuration from a file and uses it to generate and visualize orthogonal vectors.

\subsubsection{Arrowhead Matrix Examples}

\paragraph{Basic Arrowhead Usage}

\begin{lstlisting}[language=bash]
python -m generalized.main arrowhead
\end{lstlisting}

This command generates and analyzes arrowhead matrices with default parameters (4x4 matrix, 72 theta steps).

\paragraph{Customizing Matrix Generation}

\begin{lstlisting}[language=bash]
python -m generalized.main arrowhead --r0 1 1 1 --d 0.8 --theta-steps 36 --size 6
\end{lstlisting}

This command generates and analyzes 6x6 arrowhead matrices with custom parameters (origin at [1, 1, 1], d=0.8, 36 theta steps).

\paragraph{Using Perfect Circle Generation}

\begin{lstlisting}[language=bash]
python -m generalized.main arrowhead --perfect --theta-steps 12
\end{lstlisting}

This command generates and analyzes arrowhead matrices using the perfect circle generation method with 12 theta steps.

\paragraph{Only Creating Plots from Existing Results}

\begin{lstlisting}[language=bash]
python -m generalized.main arrowhead --plot-only --output-dir my_results
\end{lstlisting}

This command only creates plots from existing results in the specified directory.

\paragraph{Loading Existing Results and Creating Plots}

\begin{lstlisting}[language=bash]
python -m generalized.main arrowhead --load-only --output-dir my_results
\end{lstlisting}

This command loads existing results from the specified directory and creates plots.

\subsubsection{Saving a Configuration File}

\begin{lstlisting}[language=bash]
python -m generalized.main vector --origin 1 1 1 --d 2.0 --theta 1.047 --save-config config.json
\end{lstlisting}

This command generates and visualizes orthogonal vectors with custom parameters and saves the configuration to a file.

\subsection{Using the Arrowhead Matrix Analyzer as a Python Module}

The following examples show how to use the ArrowheadMatrixAnalyzer class directly in Python code:

\subsubsection{Basic Usage}

\begin{lstlisting}[language=Python]
import numpy as np
from example_use.arrowhead_matrix.arrowhead import ArrowheadMatrixAnalyzer

# Create an analyzer with default parameters
analyzer = ArrowheadMatrixAnalyzer()

# Generate matrices, calculate eigenvalues/eigenvectors, and create plots
analyzer.run_analysis()
\end{lstlisting}

\subsubsection{Customizing Matrix Generation}

\begin{lstlisting}[language=Python]
import numpy as np
from example_use.arrowhead_matrix.arrowhead import ArrowheadMatrixAnalyzer

# Create an analyzer with custom parameters
analyzer = ArrowheadMatrixAnalyzer(
    R_0=(1, 1, 1),           # Origin vector
    d=0.8,                   # Distance parameter
    theta_start=0,           # Starting theta value
    theta_end=2*np.pi,       # Ending theta value
    theta_steps=36,          # Number of theta steps
    coupling_constant=0.2,   # Coupling constant
    omega=1.0,               # Angular frequency
    matrix_size=6,           # Matrix size
    perfect=True,            # Use perfect circle generation
    output_dir='./results'   # Output directory
)

# Generate matrices, calculate eigenvalues/eigenvectors, and create plots
analyzer.run_analysis()
\end{lstlisting}

\subsubsection{Loading Existing Results}

\begin{lstlisting}[language=Python]
from example_use.arrowhead_matrix.arrowhead import ArrowheadMatrixAnalyzer

# Create an analyzer with the same output directory as previous runs
analyzer = ArrowheadMatrixAnalyzer(output_dir='./results')

# Load existing results and create plots
analyzer.load_results()
analyzer.create_plots()
\end{lstlisting}

\subsubsection{Accessing Eigenvalues and Eigenvectors}

\begin{lstlisting}[language=Python]
import numpy as np
from example_use.arrowhead_matrix.arrowhead import ArrowheadMatrixAnalyzer

# Create an analyzer with custom parameters
analyzer = ArrowheadMatrixAnalyzer(matrix_size=4, theta_steps=12)

# Generate matrices and calculate eigenvalues/eigenvectors
analyzer.generate_matrices()
analyzer.calculate_eigenvalues()

# Access eigenvalues and eigenvectors
for i, theta in enumerate(analyzer.theta_values):
    print(f"Theta = {theta:.4f} radians:")
    print(f"  Eigenvalues: {analyzer.eigenvalues[i]}")
    print(f"  Eigenvectors shape: {analyzer.eigenvectors[i].shape}")
    
    # Access the matrix for this theta value
    matrix = analyzer.matrices[i]
    print(f"  Matrix shape: {matrix.shape}")
\end{lstlisting}

\subsection{Complete Example Script}

The following is a complete example script that demonstrates various features of the package:

\begin{lstlisting}[language=Python]
import numpy as np
import math
import matplotlib.pyplot as plt
from mpl_toolkits.mplot3d import Axes3D
from generalized import create_orthogonal_vectors, check_orthogonality, plot_vectors, VectorConfig

def main():
    # Create configurations for different examples
    configs = [
        VectorConfig(
            origin=[0, 0, 0],
            d=1.0,
            theta=math.pi / 4,
            plot_type="3d",
            title="Default Configuration",
            save_path="default.png"
        ),
        VectorConfig(
            origin=[1, 1, 1],
            d=2.0,
            theta=math.pi / 3,
            plot_type="3d",
            title="Alternative Configuration",
            save_path="alternative.png"
        ),
        VectorConfig(
            origin=[1, 2, 3],
            d=1.5,
            num_points=36,
            perfect=True,
            plot_type="3d",
            title="Perfect Orthogonal Circle",
            save_path="perfect_circle.png"
        )
    ]
    
    # Process each configuration
    for i, config in enumerate(configs):
        print(f"Processing configuration {i+1}/{len(configs)}")
        
        # Generate orthogonal vectors
        if hasattr(config, 'perfect') and config.perfect:
            # Generate perfect orthogonal circle
            vectors = create_orthogonal_vectors(
                origin=config.origin,
                d=config.d,
                num_points=config.num_points,
                perfect=True
            )
            
            # Verify perfect circle properties
            distances = np.array([np.linalg.norm(v - config.origin) for v in vectors])
            unit_111 = np.array([1, 1, 1]) / np.sqrt(3)
            dot_products = np.array([np.abs(np.dot(v - config.origin, unit_111)) for v in vectors])
            
            print(f"  Mean distance from origin: {np.mean(distances)}")
            print(f"  Standard deviation of distances: {np.std(distances)}")
            print(f"  Maximum dot product with (1,1,1): {np.max(dot_products)}")
            
            # Custom 3D plot for perfect circle with enhanced visualization
            if config.plot_type == "3d":
                fig = plt.figure(figsize=(10, 8))
                ax = fig.add_subplot(111, projection='3d')
                
                # Apply enhanced axis styling
                from generalized.visualization import setup_enhanced_3d_axes
                setup_enhanced_3d_axes(ax, vectors)
                
                # Plot the circle
                ax.scatter(vectors[:, 0], vectors[:, 1], vectors[:, 2], label='Perfect Circle')
                
                # Plot the origin
                ax.scatter(config.origin[0], config.origin[1], config.origin[2], 
                          color='red', s=100, marker='o', label='Origin R_0')
                
                # Plot the x=y=z line
                line = np.array([[-1, -1, -1], [7, 7, 7]])
                ax.plot(line[:, 0], line[:, 1], line[:, 2], 'r--', label='x=y=z line')
                
                ax.set_title(config.title)
                ax.legend()
                
                plt.savefig(config.save_path)
                print(f"  Plot saved to {config.save_path}")
            else:
                # Use standard plot_vectors for other plot types
                plot_vectors(
                    vectors,
                    origin=config.origin,
                    plot_type=config.plot_type,
                    title=config.title,
                    show_plot=False,
                    save_path=config.save_path
                )
                print(f"  Plot saved to {config.save_path}")
        else:
            # Generate standard orthogonal vectors
            vectors = create_orthogonal_vectors(
                origin=config.origin,
                d=config.d,
                theta=config.theta
            )
            
            # Check orthogonality
            is_orthogonal = check_orthogonality(vectors, origin=config.origin)
            print(f"  Vectors are orthogonal: {is_orthogonal}")
            
            # Plot vectors
            plot_vectors(
                vectors,
                origin=config.origin,
                plot_type=config.plot_type,
                title=config.title,
                show_plot=False,
                save_path=config.save_path
            )
            print(f"  Plot saved to {config.save_path}")
        
        # Save configuration
        config_file = f"config{i+1}.json"
        config.save_to_file(config_file)
        print(f"  Configuration saved to {config_file}")
    
    # Show all plots
    plt.show()

if __name__ == "__main__":
    main()
\end{lstlisting}

This script creates three different configurations, generates orthogonal vectors for each, checks their orthogonality, plots them, and saves both the plots and configurations to files. Finally, it displays all the plots.
