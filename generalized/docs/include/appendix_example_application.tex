\newpage
\section{Example Application: Arrowhead Matrix Visualization}

This appendix contains the source code for the arrowhead matrix visualization application, which demonstrates how the orthogonal vector generation techniques can be applied to visualize eigenvalues and eigenvectors of arrowhead matrices with a coupling value of 0.1.

\subsection{generate\_arrowhead\_matrix.py}

\begin{lstlisting}[language=Python]
#!/usr/bin/env python3
"""
Generate arrowhead matrices based on orthogonal vectors.

This script generates arrowhead matrices where:
- The first diagonal element (D_00) is the sum of all VX potentials plus h*\omega
- The rest of the diagonal elements are D_ii = VXX + VA(R[i-1]) - VX(R[i-1])
- The off-diagonal elements are coupling constants
"""

import sys
import os
import numpy as np
import matplotlib.pyplot as plt
from scipy.constants import hbar  # Reduced Planck constant

# Add the parent directory to the path so we can import the modules
sys.path.append(os.path.abspath(os.path.join(os.path.dirname(__file__), '../..')))
from vector_utils import create_perfect_orthogonal_vectors, generate_R_vector


class ArrowheadMatrix:
    """
    Class to generate and manipulate arrowhead matrices based on a single orthogonal vector.
    """
    
    def __init__(self, R_0=(0, 0, 0), d=0.5, theta=0, 
                 coupling_constant=0.1, omega=1.0, perfect=True, matrix_size=4):
        """
        Initialize the ArrowheadMatrix generator for a single theta value.
        
        Parameters:
        -----------
        R_0 : tuple
            Origin vector (x, y, z)
        d : float
            Distance parameter
        theta : float
            Theta value in radians
        coupling_constant : float
            Coupling constant for off-diagonal elements
        omega : float
            Angular frequency for the energy term h*\omega
        perfect : bool
            Whether to use perfect circle generation method
        matrix_size : int
            Size of the matrix to generate (default 4x4)
        """
        self.R_0 = np.array(R_0)
        self.d = d
        self.theta = theta
        self.coupling_constant = coupling_constant
        self.omega = omega
        self.perfect = perfect
        self.matrix_size = matrix_size
        
        # Generate the R vector for this theta
        if perfect:
            self.R_vector = create_perfect_orthogonal_vectors(self.R_0, self.d, theta)
        else:
            self.R_vector = generate_R_vector(self.R_0, self.d, theta)
\end{lstlisting}

\subsection{generate\_4x4\_arrowhead.py}

\begin{lstlisting}[language=Python]
#!/usr/bin/env python3
"""
Generate a 4x4 arrowhead matrix based on orthogonal vectors.

This script generates a 4x4 arrowhead matrix where:
- The first diagonal element (D_00) is the sum of all VX potentials plus h*\omega
- The rest of the diagonal elements follow the pattern:
  D_11 = V_a(R0) + V_x(R1) + V_x(R2)
  D_22 = V_x(R0) + V_a(R1) + V_x(R2)
  D_33 = V_x(R0) + V_x(R1) + V_a(R2)
- The off-diagonal elements are coupling constants
"""

import sys
import os
import numpy as np
import matplotlib.pyplot as plt
from scipy.constants import hbar  # Reduced Planck constant
from scipy import linalg
from mpl_toolkits.mplot3d import Axes3D

# Add the parent directory to the path so we can import the modules
sys.path.append(os.path.abspath(os.path.join(os.path.dirname(__file__), '../..')))
from vector_utils import create_perfect_orthogonal_vectors, generate_R_vector


class ArrowheadMatrix4x4:
    """
    Class to generate and manipulate a 4x4 arrowhead matrix based on a single orthogonal vector.
    """
    
    def __init__(self, R_0=(0, 0, 0), d=0.5, theta=0, 
                 coupling_constant=0.1, omega=1.0, perfect=True):
        """
        Initialize the ArrowheadMatrix4x4 generator for a single theta value.
        
        Parameters:
        -----------
        R_0 : tuple
            Origin vector (x, y, z)
        d : float
            Distance parameter
        theta : float
            Theta value in radians
        coupling_constant : float
            Coupling constant for off-diagonal elements
        omega : float
            Angular frequency for the energy term h*\omega
        perfect : bool
            Whether to use perfect circle generation method
        """
        self.R_0 = np.array(R_0)
        self.d = d
        self.theta = theta
        self.coupling_constant = coupling_constant
        self.omega = omega
        self.perfect = perfect
        
        # Generate the R vector for this theta
        if perfect:
            self.R_vector = create_perfect_orthogonal_vectors(self.R_0, self.d, theta)
        else:
            self.R_vector = generate_R_vector(self.R_0, self.d, theta)
        
        # Matrix size is 4x4
        self.matrix_size = 4
\end{lstlisting}

\subsection{plot\_improved.py}

\begin{lstlisting}[language=Python]
#!/usr/bin/env python3
"""
Script to create improved plots for eigenvalues and eigenvectors:
1. 2D plots of eigenvalue vs. theta for each eigenvalue
2. 3D plots of eigenvectors without text labels
3. No connecting lines between points in any plots
4. Organizes output files into appropriate subdirectories
"""

import os
import sys
import shutil
import glob
import numpy as np
import matplotlib.pyplot as plt
from mpl_toolkits.mplot3d import Axes3D

def organize_results_directory(results_dir):
    """
    Organize the results directory by file type.
    
    Parameters:
    -----------
    results_dir : str
        Path to the results directory
    
    Returns:
    --------
    dict
        Dictionary mapping file extensions to subdirectories
    """
    # Create subdirectories if they don't exist
    subdirs = {
        'png': os.path.join(results_dir, 'plots'),
        'txt': os.path.join(results_dir, 'text'),
        'csv': os.path.join(results_dir, 'csv'),
        'npy': os.path.join(results_dir, 'numpy')
    }
    
    for subdir in subdirs.values():
        os.makedirs(subdir, exist_ok=True)
    
    # Move files to appropriate subdirectories
    for file_path in glob.glob(os.path.join(results_dir, '*')):
        # Skip directories
        if os.path.isdir(file_path):
            continue
        
        # Get file extension
        _, ext = os.path.splitext(file_path)
        ext = ext.lstrip('.')
        
        # Skip if extension not in our list
        if ext not in subdirs:
            continue
        
        # Get destination directory
        dest_dir = subdirs[ext]
        
        # Get filename
        filename = os.path.basename(file_path)
        
        # Move file to destination directory
        dest_path = os.path.join(dest_dir, filename)
        shutil.move(file_path, dest_path)
        print(f"Moved {filename} to {os.path.relpath(dest_path, results_dir)}")
    
    return subdirs

def get_file_path(results_dir, filename, file_type=None):
    """
    Get the appropriate path for a file based on its type.
    
    Parameters:
    -----------
    results_dir : str
        Path to the results directory
    filename : str
        Name of the file
    file_type : str, optional
        Type of the file (extension without the dot)
        If None, will be determined from the filename
    
    Returns:
    --------
    str
        Path to the file
    """
    # Create the directory structure if it doesn't exist
    subdirs = {
        'png': os.path.join(results_dir, 'plots'),
        'txt': os.path.join(results_dir, 'text'),
        'csv': os.path.join(results_dir, 'csv'),
        'npy': os.path.join(results_dir, 'numpy')
    }
    
    for subdir in subdirs.values():
        os.makedirs(subdir, exist_ok=True)
    
    # Determine file type if not provided
    if file_type is None:
        _, ext = os.path.splitext(filename)
        file_type = ext.lstrip('.')
    
    # Get the appropriate directory
    if file_type in subdirs:
        return os.path.join(subdirs[file_type], filename)
    else:
        return os.path.join(results_dir, filename)

def plot_eigenvalues_2d(theta_values, all_eigenvalues, output_dir="./results"):
    """
    Create 2D plots of eigenvalue vs. theta for each eigenvalue.
    
    Parameters:
    -----------
    theta_values : list of float
        List of theta values in radians
    all_eigenvalues : list of numpy.ndarray
        List of eigenvalues for each theta
    output_dir : str
        Directory to save the plots
    """
    # Convert theta values to degrees for display
    theta_values_deg = np.degrees(theta_values)
    
    # Colors for different eigenvalues
    colors = ['r', 'g', 'b', 'purple']
    
    # Create a separate 2D plot for each eigenvalue
    for ev_idx in range(4):
        plt.figure(figsize=(10, 6))
        
        # Extract the eigenvalues for this index
        ev_values = [eigenvalues[ev_idx] for eigenvalues in all_eigenvalues]
        
        # Plot eigenvalue vs. theta
        plt.scatter(theta_values_deg, ev_values, c=colors[ev_idx], s=50)
        
        # Set labels and title
        plt.xlabel('Theta (degrees)')
        plt.ylabel('Eigenvalue')
        plt.title(f'Eigenvalue {ev_idx} vs. Theta')
        
        # Add grid
        plt.grid(True, alpha=0.3)
        
        # Save the plot to the plots subdirectory
        plot_path = get_file_path(output_dir, f"eigenvalue_{ev_idx}_2d.png", "png")
        plt.savefig(plot_path, dpi=300, bbox_inches='tight')
        plt.close()
    
    # Create a combined 2D plot with all eigenvalues
    plt.figure(figsize=(12, 8))
    
    # Plot each eigenvalue series
    for ev_idx in range(4):
        ev_values = [eigenvalues[ev_idx] for eigenvalues in all_eigenvalues]
        plt.scatter(theta_values_deg, ev_values, c=colors[ev_idx], s=50, label=f'Eigenvalue {ev_idx}')
    
    # Set labels and title
    plt.xlabel('Theta (degrees)')
    plt.ylabel('Eigenvalue')
    plt.title('All Eigenvalues vs. Theta')
    
    # Add legend and grid
    plt.legend()
    plt.grid(True, alpha=0.3)
    
    # Save the plot to the plots subdirectory
    plot_path = get_file_path(output_dir, "all_eigenvalues_2d.png", "png")
    plt.savefig(plot_path, dpi=300, bbox_inches='tight')
    plt.close()

def plot_eigenvectors_no_labels(theta_values, all_eigenvectors, output_dir="./results"):
    """
    Plot the eigenvector endpoints in 3D space without text labels.
    
    Parameters:
    -----------
    theta_values : list of float
        List of theta values in radians
    all_eigenvectors : list of numpy.ndarray
        List of eigenvectors for each theta
    output_dir : str
        Directory to save the plots
    """
    # Create a figure for all eigenvector endpoints
    fig = plt.figure(figsize=(14, 12))
    ax = fig.add_subplot(111, projection='3d')
    
    # Convert theta values to degrees for display
    theta_values_deg = np.degrees(theta_values)
    
    # Colors for different theta values
    cmap = plt.cm.viridis
    theta_colors = [cmap(i/len(theta_values)) for i in range(len(theta_values))]
    
    # Markers for different eigenvectors
    markers = ['o', 's', '^', 'd']  # circle, square, triangle, diamond
    
    # Plot all eigenvector endpoints
    for i, (theta, eigenvectors) in enumerate(zip(theta_values_deg, all_eigenvectors)):
        for ev_idx in range(4):
            eigenvector = eigenvectors[:, ev_idx]
            ax.scatter(eigenvector[0], eigenvector[1], eigenvector[2], 
                       c=[theta_colors[i]], marker=markers[ev_idx], s=100, 
                       alpha=0.8)
    
    # Set labels and title
    ax.set_xlabel('X Component')
    ax.set_ylabel('Y Component')
    ax.set_zlabel('Z Component')
    ax.set_title('Eigenvector Endpoints (No Labels)')
    
    # Create a custom colorbar for theta values
    sm = plt.cm.ScalarMappable(cmap=cmap, norm=plt.Normalize(vmin=0, vmax=360))
    sm.set_array([])
    cbar = plt.colorbar(sm, ax=ax, shrink=0.7, aspect=20)
    cbar.set_label('Theta (degrees)')
    
    # Create a custom legend for eigenvector indices
    from matplotlib.lines import Line2D
    legend_elements = [
        Line2D([0], [0], marker=markers[i], color='w', markerfacecolor='gray', 
               markersize=10, label=f'Eigenvector {i}')
        for i in range(4)
    ]
    ax.legend(handles=legend_elements, loc='upper right')
    
    # Save the plot to the plots subdirectory
    plot_path = get_file_path(output_dir, "eigenvectors_no_labels.png", "png")
    plt.savefig(plot_path, dpi=300, bbox_inches='tight')
    plt.close()
    
    # Create individual plots for each eigenvector
    for ev_idx in range(4):
        fig = plt.figure(figsize=(12, 10))
        ax = fig.add_subplot(111, projection='3d')
        
        # Plot eigenvector endpoints for this eigenvector index
        for i, (theta, eigenvectors) in enumerate(zip(theta_values_deg, all_eigenvectors)):
            eigenvector = eigenvectors[:, ev_idx]
            ax.scatter(eigenvector[0], eigenvector[1], eigenvector[2], 
                       c=[theta_colors[i]], marker='o', s=100)
        
        # Set labels and title
        ax.set_xlabel('X Component')
        ax.set_ylabel('Y Component')
        ax.set_zlabel('Z Component')
        ax.set_title(f'Eigenvector {ev_idx} Endpoints (No Labels)')
        
        # Add colorbar for theta values
        sm = plt.cm.ScalarMappable(cmap=cmap, norm=plt.Normalize(vmin=0, vmax=360))
        sm.set_array([])
        cbar = plt.colorbar(sm, ax=ax, shrink=0.7, aspect=20)
        cbar.set_label('Theta (degrees)')
        
        # Save the plot to the plots subdirectory
        plot_path = get_file_path(output_dir, f"eigenvector_{ev_idx}_no_labels.png", "png")
        plt.savefig(plot_path, dpi=300, bbox_inches='tight')
        plt.close()

def main():
    """
    Main function to load saved eigenvalues and eigenvectors and create improved plots.
    """
    # Define the results directory
    results_dir = os.path.join(os.path.dirname(os.path.abspath(__file__)), 'results')
    os.makedirs(results_dir, exist_ok=True)
    
    # Organize the results directory
    print("Organizing results directory...")
    organize_results_directory(results_dir)
    
    # Find all eigenvalue and eigenvector files
    numpy_dir = os.path.join(results_dir, 'numpy')
    eigenvalue_files = sorted(glob.glob(os.path.join(numpy_dir, 'eigenvalues_theta_*.npy')))
    eigenvector_files = sorted(glob.glob(os.path.join(numpy_dir, 'eigenvectors_theta_*.npy')))
    
    # Check if we have files to process
    if not eigenvalue_files or not eigenvector_files:
        print("No eigenvalue or eigenvector files found. Please run generate_plots_no_connections.py first.")
        return
    
    print(f"Found {len(eigenvalue_files)} sets of eigenvalues and eigenvectors.")
    
    # Load the eigenvalues and eigenvectors
    all_eigenvalues = []
    all_eigenvectors = []
    theta_values = []
    
    for ev_file, evec_file in zip(eigenvalue_files, eigenvector_files):
        # Extract theta index from filename
        theta_idx = int(os.path.basename(ev_file).split('_')[-1].split('.')[0])
        
        # Calculate theta in radians (assuming 5-degree increments)
        theta = np.radians(theta_idx * 5)
        theta_values.append(theta)
        
        # Load eigenvalues and eigenvectors
        eigenvalues = np.load(ev_file)
        eigenvectors = np.load(evec_file)
        
        all_eigenvalues.append(eigenvalues)
        all_eigenvectors.append(eigenvectors)
    
    # Sort by theta value
    sorted_indices = np.argsort(theta_values)
    theta_values = [theta_values[i] for i in sorted_indices]
    all_eigenvalues = [all_eigenvalues[i] for i in sorted_indices]
    all_eigenvectors = [all_eigenvectors[i] for i in sorted_indices]
    
    # Create the improved plots
    print("Creating improved plots...")
    
    print("Creating 2D eigenvalue plots...")
    plot_eigenvalues_2d(theta_values, all_eigenvalues, results_dir)
    
    print("Creating eigenvector plots without labels...")
    plot_eigenvectors_no_labels(theta_values, all_eigenvectors, results_dir)
    
    # List the created plots
    plots_dir = os.path.join(results_dir, 'plots')
    plot_files = sorted(glob.glob(os.path.join(plots_dir, '*.png')))
    
    print("Plots created successfully:")
    for plot_file in plot_files:
        print(f"  - {plot_file}")

if __name__ == "__main__":
    main()
\end{lstlisting}

\subsection{file\_utils.py}

\begin{lstlisting}[language=Python]
#!/usr/bin/env python3
"""
Utility functions for file operations in the arrowhead matrix visualization.
"""

import os
import glob
import shutil

def organize_results_directory(results_dir):
    """
    Organize the results directory by file type.
    
    Parameters:
    -----------
    results_dir : str
        Path to the results directory
    
    Returns:
    --------
    dict
        Dictionary mapping file extensions to subdirectories
    """
    # Create subdirectories if they don't exist
    subdirs = {
        'png': os.path.join(results_dir, 'plots'),
        'txt': os.path.join(results_dir, 'text'),
        'csv': os.path.join(results_dir, 'csv'),
        'npy': os.path.join(results_dir, 'numpy')
    }
    
    for subdir in subdirs.values():
        os.makedirs(subdir, exist_ok=True)
    
    # Move files to appropriate subdirectories
    for file_path in glob.glob(os.path.join(results_dir, '*')):
        # Skip directories
        if os.path.isdir(file_path):
            continue
        
        # Get file extension
        _, ext = os.path.splitext(file_path)
        ext = ext.lstrip('.')
        
        # Skip if extension not in our list
        if ext not in subdirs:
            continue
        
        # Get destination directory
        dest_dir = subdirs[ext]
        
        # Get filename
        filename = os.path.basename(file_path)
        
        # Move file to destination directory
        dest_path = os.path.join(dest_dir, filename)
        shutil.move(file_path, dest_path)
        print(f"Moved {filename} to {os.path.relpath(dest_path, results_dir)}")
    
    return subdirs

def get_file_path(results_dir, filename, file_type=None):
    """
    Get the appropriate path for a file based on its type.
    
    Parameters:
    -----------
    results_dir : str
        Path to the results directory
    filename : str
        Name of the file
    file_type : str, optional
        Type of the file (extension without the dot)
        If None, will be determined from the filename
    
    Returns:
    --------
    str
        Path to the file
    """
    # Create the directory structure if it doesn't exist
    subdirs = {
        'png': os.path.join(results_dir, 'plots'),
        'txt': os.path.join(results_dir, 'text'),
        'csv': os.path.join(results_dir, 'csv'),
        'npy': os.path.join(results_dir, 'numpy')
    }
    
    for subdir in subdirs.values():
        os.makedirs(subdir, exist_ok=True)
    
    # Determine file type if not provided
    if file_type is None:
        _, ext = os.path.splitext(filename)
        file_type = ext.lstrip('.')
    
    # Get the appropriate directory
    if file_type in subdirs:
        return os.path.join(subdirs[file_type], filename)
    else:
        return os.path.join(results_dir, filename)
\end{lstlisting}
