\documentclass{article}
\usepackage{amsmath}
\usepackage{amssymb}
\usepackage{bm}

\begin{document}

\section{Hamiltonian Formulation}
The Hamiltonian $H(\theta)$ is a 4x4 arrowhead matrix defined as:
\begin{equation}
H(\theta) = \begin{pmatrix}
E_0(\theta) & c & c & c \\
c & E_1(\theta) & 0 & 0 \\
c & 0 & E_2(\theta) & 0 \\
c & 0 & 0 & E_3(\theta)
\end{pmatrix}
\end{equation}
where $E_0(\theta) = \hbar\omega + \sum_{i=0}^2 V_x^{(i)}(\theta)$ and $E_{i+1}(\theta) = E_0(\theta) + V_a^{(i)}(\theta) - V_x^{(i)}(\theta)$ for $i=0,1,2$.
Here $\omega$ is a frequency parameter, $c = 0.2$ is a fixed coupling constant, and $V_x$ and $V_a$ are potential terms that depend on the parameter vector $\bm{R}(\theta)$. These potential terms are defined as follows:
\begin{align}
V_x^{(i)}(R_i) &= a \cdot (R_i)^2 + b \cdot R_i + c \\
V_a^{(i)}(R_i) &= a \cdot (R_i - x_{\text{shift}})^2 + c
\end{align}

\section{R$_{\theta}$ generation}
The R$_{\theta}$ vector traces a perfect circle orthogonal to the $x=y=z$ line using the \texttt{create\_perfect\_orthogonal\_vectors} function from the Arrowhead/generalized package.


\section{Berry Connection}
The Berry connection $A(\theta)$ is calculated using:
\begin{equation}
A_{n}(\theta_j) = \langle n(\theta_j) | i \partial_\theta | n(\theta_j) \rangle
\end{equation}
where $|n(\theta_j)\rangle$ are the eigenstates of $H(\theta_j)$.

\section{Berry Phase}
The Berry phase $\gamma_n$ for state $n$ is obtained by integrating the Berry connection:
\begin{equation}
\gamma_n = \int_0^{2\pi} A_n(\theta) d\theta
\end{equation}

\section{Verification}
We verify the eigenvalue equation $H(\theta)|n(\theta)\rangle = E_n(\theta)|n(\theta)\rangle$ by comparing:
\begin{equation}
H(\theta)|n(\theta)\rangle \quad \text{vs} \quad E_n(\theta)|n(\theta)\rangle
\end{equation}

\section{Visualization}
For each state $n$, we plot:
\begin{itemize}
\item Magnitude of $H(\theta)|n(\theta)\rangle$ and $E_n(\theta)|n(\theta)\rangle$
\item Real and imaginary components separately
\item All four vector components
\end{itemize}

\end{document}
